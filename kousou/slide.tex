\RequirePackage{plautopatch}
\documentclass[11pt,aspectratio=169,xcolor=dvipsnames,table,dvipdfmx]{beamer}
\usepackage{bxdpx-beamer} % dvipdfmxなので必要
\usepackage{appendixnumberbeamer}

%Beamerの設定
\usetheme{Boadilla}

%Beamerフォント設定
\usefonttheme{professionalfonts} % Be professional!
\usepackage[T1]{fontenc}
\usepackage{mlmodern}  % 太いComputer Modern
% MLmodernのバグを修正: cf. https://tex.stackexchange.com/questions/646333/size-of-integral-symbol-in-section-header-with-mlmodern
\DeclareFontFamily{OMX}{mlmex}{}
\DeclareFontShape{OMX}{mlmex}{m}{n}{%
   <->mlmex10%
   }{} 
\usepackage{newtxtext} % 数式以外をTXフォントで上書き
\usepackage[deluxe,uplatex]{otf} % 日本語多ウェイト化
\usepackage{physics,bm}
\usepackage{mhchem}
\usepackage{amsmath,amsfonts,amssymb,mathtools,amsthm}
\DeclareMathOperator{\sgn}{sgn}
\renewcommand{\familydefault}{\sfdefault}  % 英文をサンセリフ体に
\renewcommand{\kanjifamilydefault}{\gtdefault}  % 日本語をゴシック体に
\usefonttheme{structurebold} % タイトル部を太字
\setbeamerfont{alerted text}{series=\bfseries} % Alertを太字
\setbeamerfont{section in toc}{series=\mdseries} % 目次は太字にしない
\setbeamerfont{frametitle}{size=\Large} % フレームタイトル文字サイズ
\setbeamerfont{title}{size=\LARGE} % タイトル文字サイズ
\setbeamerfont{date}{size=\small}  % 日付文字サイズ

% Babel (日本語の場合のみ・英語の場合は不要)
\uselanguage{japanese}
\languagepath{japanese}
\deftranslation[to=japanese]{Theorem}{定理}
\deftranslation[to=japanese]{Lemma}{補題}
\deftranslation[to=japanese]{Example}{例}
\deftranslation[to=japanese]{Examples}{例}
\deftranslation[to=japanese]{Definition}{定義}
\deftranslation[to=japanese]{Definitions}{定義}
\deftranslation[to=japanese]{Problem}{問題}
\deftranslation[to=japanese]{Solution}{解}
\deftranslation[to=japanese]{Fact}{事実}
\deftranslation[to=japanese]{Proof}{証明}
\def\proofname{証明}

%Beamer色設定
\definecolor{UniBlue}{RGB}{0,150,200} 
\definecolor{AlertOrange}{RGB}{255,76,0}
\definecolor{AlmostBlack}{RGB}{38,38,38}
\setbeamercolor{normal text}{fg=AlmostBlack}  % 本文カラー
\setbeamercolor{structure}{fg=UniBlue} % 見出しカラー
\setbeamercolor{block title}{fg=UniBlue!50!black} % ブロック部分タイトルカラー
\setbeamercolor{alerted text}{fg=AlertOrange} % \alert 文字カラー
\mode<beamer>{
    \definecolor{BackGroundGray}{RGB}{254,254,254}
    \setbeamercolor{background canvas}{bg=BackGroundGray} % スライドモードのみ背景をわずかにグレーにする
}

%フラットデザイン化
\setbeamertemplate{blocks}[rounded] % Blockの影を消す
\useinnertheme{circles} % 箇条書きをシンプルに
\setbeamertemplate{navigation symbols}{} % ナビゲーションシンボルを消す
\setbeamertemplate{footline}[frame number] % フッターはスライド番号のみ
\setbeamerfont{footline}{size=\small}    % フッターを大きくした

%タイトルページ
\setbeamertemplate{title page}{%
    \vspace{2.5em}
    {\usebeamerfont{title} \usebeamercolor[fg]{title} \inserttitle \par}
    {\usebeamerfont{subtitle}\usebeamercolor[fg]{subtitle}\insertsubtitle \par}
    \vspace{1.5em}
    \begin{flushright}
        \usebeamerfont{author}\insertauthor\par
        \usebeamerfont{institute}\insertinstitute \par
        \vspace{3em}
        \usebeamerfont{date}\insertdate\par
      \end{flushright}
      \vspace{-6em}
      \begin{flushleft}
        \usebeamercolor[fg]{titlegraphic}\inserttitlegraphic
      \end{flushleft}
}

% Algorithm系
\usepackage{algorithm}
\usepackage[noend]{algorithmic}
\algsetup{linenosize=\color{fg!50}\footnotesize}
\renewcommand\algorithmicdo{:}
\renewcommand\algorithmicthen{:}
\renewcommand\algorithmicrequire{\textbf{Input:}}
\renewcommand\algorithmicensure{\textbf{Output:}}

% 定理
\theoremstyle{definition}
\newenvironment{mythm}{\begin{alertblock}{定理}}{\end{alertblock}} %自分の結果は赤色で表示

\AtBeginSection[]{
    \frame{\tableofcontents[currentsection, hideallsubsections]} %目次スライド
}
\usepackage[absolute,overlay]{textpos}
\usepackage{array} % needed for \arraybackslash
\usepackage{adjustbox} % for \adjincludegraphics

\usepackage{tikz}
\usetikzlibrary{calc}
\makeatletter
\chardef\zenkakuSpace=\jis"2121\relax % 和文ゴースト用全角スペース

\def\serifRoman#1{%
  \ifnum0\ifnum#1<1 1\fi\ifnum#1>12 1\fi>0 %%% 引数が 1~12 の範囲外ならエラー
    \@latex@error{\string\serifRoman: #1 must be 1..12}\@ehc
  \fi
  \zenkakuSpace\kern-1zw\relax %%% ゴースト処理(入口側)
  \begingroup
  \@tempcnta="215F\relax
  \advance\@tempcnta by #1\relax
  \kchardef\@roman@char=\ucs\@tempcnta\relax %% 今回対象のローマ数字
  \kchardef\@roman@I=\ucs"2160\relax %% ヒゲとして利用するローマ数字Ⅰ
  \@tempdima=-.132zw\relax
  \@tempdimb=.134zw\relax
  \@tempdimc=1zw\relax
  \setbox1\hbox to 1zw{\yoko\hfill
    \begin{tikzpicture}[baseline=(A.base),inner sep=0pt, outer sep=0pt]%
    \path[use as bounding box] (-.5\@tempdimc, -.5\@tempdimc) rectangle (.5\@tempdimc, .5\@tempdimc);
    \node (A) {\@roman@char};
    \ifcase#1
      \or % Ⅰ
        \@serifRoman@a{.45}{0}% 上ヒゲ
        \@serifRoman@b{.45}{0}% 下ヒゲ
      \or % Ⅱ
        \@serifRoman@a{.80}{0}% 上ヒゲ
        \@serifRoman@b{.80}{0}% 下ヒゲ
      \or % Ⅲ
        \@serifRoman@a{1.10}{0}% 上ヒゲ
        \@serifRoman@b{1.10}{0}% 下ヒゲ
      \or % Ⅳ
        \@serifRoman@a{.60}{-.24}% 左上ヒゲ
        \@serifRoman@a{.30}{.38}%% 右上ヒゲ
        \@serifRoman@b{.30}{-.35}% 左下ヒゲ
      \or % Ⅴ
        \@serifRoman@a{.30}{-.28}% 左上ヒゲ
        \@serifRoman@a{.30}{.28}%% 右上ヒゲ
      \or % Ⅵ
        \@serifRoman@a{.30}{-.38}% 左上ヒゲ
        \@serifRoman@a{.60}{.24}%% 右上ヒゲ
        \@serifRoman@b{.30}{.34}%% 右下ヒゲ
      \or % Ⅶ
        \@serifRoman@a{.30}{-.41}% 左上ヒゲ
        \@serifRoman@a{.75}{.20}%% 右上ヒゲ
        \@serifRoman@b{.52}{.28}%% 右下ヒゲ
      \or % Ⅷ
        \@serifRoman@a{.30}{-.41}% 左上ヒゲ
        \@serifRoman@a{.82}{.18}%% 右上ヒゲ
        \@serifRoman@b{.65}{.245}% 右下ヒゲ
      \or % Ⅸ
        \@serifRoman@a{.25}{.33}%% 右上ヒゲ
        \@serifRoman@a{.55}{-.22}% 左上ヒゲ
        \@serifRoman@b{.55}{-.22}% 左下ヒゲ
        \@serifRoman@b{.25}{.36}%% 右下ヒゲ
      \or % Ⅹ
        \@serifRoman@a{.25}{.23}%% 右上ヒゲ
        \@serifRoman@a{.25}{-.23}% 左上ヒゲ
        \@serifRoman@b{.25}{-.26}% 左下ヒゲ
        \@serifRoman@b{.25}{.26}%% 右下ヒゲ
      \or % Ⅺ
        \@serifRoman@a{.55}{.23}%% 右上ヒゲ
        \@serifRoman@a{.25}{-.34}% 左上ヒゲ
        \@serifRoman@b{.25}{-.36}% 左下ヒゲ
        \@serifRoman@b{.55}{.23}%% 右下ヒゲ
      \or % Ⅻ
        \@serifRoman@a{.75}{.20}%% 右上ヒゲ
        \@serifRoman@a{.25}{-.38}% 左上ヒゲ
        \@serifRoman@b{.25}{-.40}% 左下ヒゲ
        \@serifRoman@b{.75}{.20}%% 右下ヒゲ
    \fi
    \end{tikzpicture}%
    \hfill}%
  \box1\relax
  \endgroup
  \kern-1zw\relax\zenkakuSpace %%% ゴースト処理(出口側)
}
\def\@serifRoman@a#1#2{%
  \node[rotate=90,xscale=.7,yscale=#1] at ($(A.north) + (#2\@tempdimc,\@tempdima)$) {\@roman@I};
}
\def\@serifRoman@b#1#2{%
  \node[rotate=90,xscale=.7,yscale=#1] at ($(A.south) + (#2\@tempdimc,\@tempdimb)$) {\@roman@I};
}
\makeatother


%タイトル
\title{軸対称Skyrme TDHFを用いた\\トンネル確率の計算}
\author{\textbf{松本侑真}}
\date{\today}
\institute{原子核理論 関澤研究室}
\titlegraphic{\includegraphics[scale=0.25]{fission_1.png}}
%\titlegraphic{\adjincludegraphics[height=.5\linewidth,valign=t]{heikinba.png}}

\begin{document}
\maketitle
%\frame{\tableofcontents[hideallsubsections]}

\begin{frame}{背景}
  \begin{columns}[t]
    \begin{column}{.5\textwidth}
      \begin{itemize}
        \item トンネル効果によって生じる現象が存在する
        \item 1次元模型では、ガモフによる$\alpha$崩壊の理論やWKB近似などがある
        \item 実際の核分裂は多粒子系のトンネル現象かつ複雑な形状の自由度があり、微視的な計算が困難
      \end{itemize}
      \vskip\baselineskip
      \begin{itemize}
        \item \textbf{本研究の目的:} 平均場理論に虚時間発展を適用し、核子自由度から微視的に多体のトンネル現象を記述すること
      \end{itemize}
  
    \end{column}
    \begin{column}{.5\textwidth}
      \adjincludegraphics[height=1\linewidth,valign=t]{tunneling.png}
    \end{column}
  \end{columns}

\end{frame}


\begin{frame}{背景}
  J.W.Negele, Nuclear Mean-Field Theory, Physics Today 38, 24(1985)では
  \begin{itemize}
    \item 平均場理論+経路積分(虚時間発展の手法)を用いて、1粒子系のトンネル現象でWKB近似の結果を再現
    \item 少数核子系(\ce{^{8}Be})の$\alpha$崩壊の計算に応用
  \end{itemize}
  を行っている。しかし、結果の定量的な評価はしておらず、詳細は述べられていない。
  \adjincludegraphics[width=1\linewidth,valign=t]{Be.png}
\end{frame}


\begin{frame}{理論的枠組み:Hartree-Fock近似}
  \begin{itemize}
    \item Hartree-Fock(HF)近似は平均場理論の近似法であり、原子核の性質(束縛エネルギーや変形度)を上手く説明できる
    \item 時間依存のHF方程式(TDHF方程式)は核子自由度から微視的に原子核ダイナミクスを記述できる
  \end{itemize}
  \begin{block}{TDHFの問題点}
    \begin{itemize}
      \item 多体波動関数が1つのSlater行列式で表されるという近似を用いている
            \begin{itemize}
              \item 複数のSlater行列式の線形結合で波動関数を表すことで、より正確な状態が得られることが知られている
            \end{itemize}
      \item 核分裂反応のような\textbf{多体のトンネル現象を記述できない}
            \begin{itemize}
              \item 平均場が1つしかなく、核分裂していない状態と核分裂している状態の重ね合わせが記述できない
            \end{itemize}
    \end{itemize}
  \end{block}
  トンネル現象を記述するためには何らかの工夫をする必要があり、そのための手法を開発することが研究の目標である。
\end{frame}

\begin{frame}{理論的枠組み:経路積分+虚時間発展}
  \begin{itemize}
    \item ハミルトニアンが$\hat{H}$である系の時間発展演算子$U$は$U=e^{-i\hat{H}t/\hbar}$である。
    \item 初期状態$\ket{i}$が終状態$\ket{f}$に遷移する確率振幅は、作用$S$を用いて経路積分で表される:
    \begin{equation}
      \bra{f}U\ket{i} = \int\mathcal{D}[q]\exp(\frac{i}{\hbar}S[q])\;\text{。}
    \end{equation}
    \item $\tau$を実数として、$t\to -i\tau$の変換をWick回転と呼ぶ。Wick回転を施した時間発展
    \begin{equation}
      \psi(\tau) = \exp(-\hat{H}\tau/\hbar)\psi(0)
    \end{equation}
    は、虚時間発展と呼ばれる。
    \item Negele(1985)によると、経路積分に虚時間発展を用いた計算を行うことで、WKB近似におけるトンネル効果の理論を再現できる。
  \end{itemize}
\end{frame}

\begin{frame}{背景:Toy model(二重井戸モデル)でのトンネル効果}
  P. McGlynn and C. Simenel, Phys. Rev. C 102, 064614 (2018)では
  \begin{itemize}
    \item Toy model(1次元の二重井戸モデル)において、$L\to R$へのトンネル確率を平均場理論+虚時間発展を用いて計算
    \begin{itemize}
      \item 2粒子系が$L$もしくは$R$の領域に束縛される場合を考えている
    \end{itemize}
    \item 平均場理論+実時間発展ではトンネル確率を正しく計算できないことの確認
  \end{itemize}
  を行っている。
  \begin{center}
  \adjincludegraphics[height=0.3\linewidth,valign=t]{niju-ido.png}
  \end{center}
\end{frame}

\begin{frame}
  \frametitle{背景:Toy model(二重井戸モデル)でのトンネル効果}
  2粒子系の二重井戸モデルのハミルトニアン$\hat{H}\in\mathcal{L}(\mathcal{H}\otimes\mathcal{H})$は、以下のように表される:
  \begin{equation}
    \hat{H} = \hat{h}_0\otimes\hat{1}+\hat{1}\otimes\hat{h}_0+\hat{v}\;\text{。}
  \end{equation}
  ここで、$\ket{-}$を励起状態とし、基底状態$\ket{+}$のエネルギーは$0$とする(二準位系)。左右の井戸に粒子が存在していることを表す状態$\ket{L},\,\ket{R}$を用いて
  \begin{equation}
    \ket{\pm} = \frac{1}{\sqrt{2}}\qty(\ket{L}\pm\ket{R})
  \end{equation}
  と表される。このとき、一粒子ハミルトン$\hat{h_0}$と、相互作用項$\hat{v}$は
  \begin{equation}
    \hat{h}_0 = \alpha\ketbra{-}\;\text{,}\quad \hat{v} = \mu(\ketbra{LL} + \ketbra{RR})
  \end{equation}
  となる。なお、時刻$t$における一般の状態$\ket{\psi(t)}$は、$\ket{L},\,\ket{R}$の重ね合わせで表される:
  \begin{equation}
    \ket{\psi(t)} = L(t)\ket{L} + R(t)\ket{R}\;\text{。}
  \end{equation}
  
\end{frame}

\begin{frame}
  \frametitle{背景:平均場理論+実時間発展を用いたToy modelのトンネル確率}
  \begin{itemize}
    \item 左の井戸に存在する粒子数の期待値$\ev{\hat{N}_{L}}$の{\color{blue}厳密解}は以下のようになる:
          \begin{equation}
            \ev{\hat{N}_{L}} = 1 + \frac{\beta-\mu}{2\beta}\cos\qty(\frac{\beta+\mu}{2}t) + 
            \frac{\beta + \mu}{2}\cos\qty(\frac{\beta-\mu}{2}t)\;\text{,}\quad \beta = \sqrt{4\alpha^2+\mu^2}\;\text{。}
          \end{equation}
    \item {\color{orange}平均場理論を用いた実時間発展の解}は、$\ev{\hat{N}_{L}} \propto 1 + \sin\theta(t)$となる。
  \end{itemize}
  \begin{columns}[t]
    \begin{column}{.45\textwidth}
      \begin{itemize}
        \item $\mu$が増加すると、平均場理論の結果は急速に厳密解からずれていく。
        \item 特に、$\abs{\mu}>2$の``強い''相互作用が働く場合は、\textbf{片方の井戸に粒子が束縛される}結果となってしまい、トンネル効果が記述できない。
      \end{itemize}
    \end{column}
    \begin{column}{.55\textwidth}
      \adjincludegraphics[width=1\linewidth,valign=t]{realtime-ev.png}
    \end{column}
  \end{columns}
\end{frame}

\begin{frame}
  \frametitle{背景:平均場理論+虚時間発展を用いたToy modelのトンネル確率}
  \begin{itemize}
    \item 平均場理論に虚時間発展(Wick回転:$t\to -i\tau$)を組み込むことで、トンネル効果を記述できることが知られている。このアプローチ方法は、WKB半古典近似と等価なものである。
    \item $\theta = \arcsin(\abs{L}^2-\abs{R}^2),\,\phi = \arg(R/L)$で定義されるパラメータ$\theta,\,\phi$を導入する。このパラメータを持つ状態の時間発展は
    \begin{equation}
      \dv{\theta}{\tau} = i\sin\phi,\quad \dv{\phi}{\tau} = -i\tan\theta\cos\phi - i\mu\sin\theta
    \end{equation}
    の式に従い、一般に複素数である。
    \item $(\theta,\,\phi)$でラベルされる状態のトンネル確率$P(\theta,\,\phi)$は、
    \begin{equation}
      P(\theta,\,\phi) = \abs{e^{iW[\theta,\,\phi]}}^2 \coloneqq \abs{\exp(\frac{i}{2}\int_0^T d\tau\, (\tan\theta(\tau)\cos\phi(\tau) + \mu\sin\theta(\tau))\sin\theta(\tau))}^2
    \end{equation}
    で与えられる。初期状態の$(\theta,\,\phi)$を定めると、Toy modelにおけるトンネル確率が平均場理論+虚時間発展の枠組みで計算できる。
  \end{itemize}
\end{frame}

\begin{frame}{背景:Toy modelのトンネル確率(まとめ)}
  \begin{columns}[t]
    \begin{column}{.5\textwidth}
      \begin{itemize}
        \item 実時間発展(上図)では、{\color{blue}厳密解}と{\color{orange}計算値}の結果が異なる
        \item 虚時間発展(下図)では、相互作用が強いほど{\color{blue}厳密解}に近づく
        \begin{itemize}
          \item {\color{blue}厳密解}はWKB近似から計算される値を用いている
          \item $\mu$は2粒子の波動関数が同じ井戸に存在する場合の引力相互作用の強さ
        \end{itemize}
        \item 下図のような結果を\textbf{現実的な原子核系}で再現することが研究の目標である。
      \end{itemize}
    \end{column}
    \begin{column}{.5\textwidth}
      \adjincludegraphics[width=1\linewidth,valign=t]{realtime-ev.png}
      \adjincludegraphics[width=1\linewidth,valign=t]{imagtime-ev.png}
    \end{column}
  \end{columns}
\end{frame}

\begin{frame}
  \frametitle{背景:核子多体系におけるトンネル確率の計算}
  \begin{itemize}
    \item P. McGlynn, Microscopic Models of Nuclear Reactions Near the Coulomb Barrier, Ph.D. thesis, Australian National University (2023)では、
    Toy modelでの手法を3次元の核子多体系へ応用しようと試みたが、失敗していることが示されている。
    \item 具体的に何を試みているのか、どこが失敗しているのかは未調査だが、系の虚時間発展における\textbf{数値計算の安定性}が問題となっているようである。
    \item 卒論では、McGlynn(2023)での手法を参考にしながら、トンネル確率の計算方法を考えていく予定である。
  \end{itemize}
\end{frame}


\begin{frame}
  \frametitle{自分の卒業研究について}
  \begin{block}{研究概要}
    \begin{enumerate}
      \item 3次元空間で軸対称性を課した原子核において、平均場理論に基づいた原子核の基底状態を求める自作プログラムを作成する。
      \item 完成したプログラムを用いて、実時間発展ではトンネル効果を再現できないことを確認する。
      \item 虚時間発展を組み込んで、平均場理論の枠組みでトンネル確率を計算することを目指す。
    \end{enumerate}
  \end{block}
  \begin{exampleblock}{研究の現状と今後の予定}
    \begin{itemize}
      \item 現在は、原子核の基底状態を求める自作プログラムの作成途中である。
      \begin{itemize}
        \item 12月末を目処に基底状態を求めるプログラムを完了させる予定である。
      \end{itemize}
      \item 基底状態のプログラムが完成した後に行うこと:
      \begin{itemize}
        \item 虚時間発展を組み込む方法を検討する(論文の手法をまとめるだけで終わるかもしれない)。
        \item プログラムを改良して原子核の実時間・虚時間発展を計算する。
      \end{itemize}
    \end{itemize}
  \end{exampleblock}

\end{frame}



\begin{frame}{研究手法:基底状態の求め方}
  \begin{itemize}
    \item Skyrme型の相互作用を入れたHartree-Fock(SHF)方程式をセルフコンシステントに解いて原子核の基底状態を求める。
    \begin{gather}
      \hat{h}_{\alpha}(\bm{r}) = U_\alpha(\bm{r}) - \div[B_\alpha(\bm{r})\grad] + i\bm{W}_\alpha(\bm{r})\cdot(\bm{\sigma}\cross\bm{\nabla})\quad\text{(一粒子ハミルトニアン)} \\
      \hat{h}_{\alpha}(\bm{r})\psi_\alpha(\bm{r}) = \varepsilon_\alpha\psi_\alpha(\bm{r})\quad\text{(SHF方程式)}
    \end{gather}
  \end{itemize}
  \begin{block}{SHFの解き方(平均場が収束するまで繰り返す)}
    \begin{enumerate}
      \item 核子の一粒子波動関数(スピノル)$\psi_{\alpha}$から、粒子数密度、スピン-軌道密度、運動エネルギー密度
      \begin{equation}
        \rho(r,z),\,\bm{J}(r,z),\,\tau(r,\,z)
      \end{equation}を計算する。
      \item 各密度に依存した平均場
      \begin{equation}
        U(r,z),\,\bm{W}(r,z),\,B(r,z)
      \end{equation}
      を計算する。
      \item 平均場で構成される一粒子ハミルトニアン$\hat{h}_{\alpha}$を用いて、スピノルの値を更新する。
    \end{enumerate}
  \end{block}
\end{frame}

\begin{frame}{研究手法:基底状態の求め方}
密度や平均場が軸対称性を持つとき、核子のスピノルは以下のように表される。
\begin{equation}
  \psi_{\alpha}(\bm{r}) =
  \begin{pmatrix}
    \psi_{\alpha}^{+}(\bm{r}) \\
    \psi_{\alpha}^{-}(\bm{r})
  \end{pmatrix}
  = 
  \begin{pmatrix}
    f^{+}_{\alpha}(r,z)e^{im_{\alpha}\phi} \\
    f^{-}_{\alpha}(r,z)e^{i(m_{\alpha}+1)\phi}
  \end{pmatrix}
\end{equation}
このスピノルによって、密度は以下のように計算できる:
\begin{gather}
  \rho(r,z) = \sum_{\alpha}\qty[\abs{f_{\alpha}^{+}(\bm{r})}^2+\abs{f_{\alpha}^{-}(\bm{r})}^2],\quad 
  \bm{J}(r,z) = \sum_{\alpha}\Im[\psi_{\alpha}^{\dagger}(\bm{r})(\bm{\nabla}\cross\bm{\sigma})\psi_{\alpha}(\bm{r})],\\
  \tau(r,z) = \sum_{\alpha}\qty[\abs{\bm{\nabla}f_{\alpha}^{+}(\bm{r})}^2+\abs{\bm{\nabla}f_{\alpha}^{-}(\bm{r})}^2]\text{。}
\end{gather}
\begin{itemize}
  \item 密度$\rho,\,\bm{J},\,\tau$が求まると、平均場$U,\,\bm{W},\,B$が求まり、一粒子ハミルトニアン$\hat{h}_{\alpha}$が求まる。
\end{itemize}
\end{frame}

\begin{frame}{研究手法:基底状態の求め方}
  本プログラムでは、``虚時間法''($\neq$虚時間発展)によって基底状態を求める。
  虚時間法では、\textbf{一粒子波動関数と平均場ポテンシャルの反復更新}によって、系の全エネルギーが最小化される。系のHamiltonianを$\hat{H}$、固有エネルギーを$E_n$とする:
  \begin{equation}
    \hat{H}\ket{\Psi_{n}}=E_{n}\ket{\Psi_{n}}\quad (E_{0}\leq E_{1}\leq E_{2}\leq\cdots)\;\text{。}
  \end{equation}
  このとき、任意の状態$\ket{\xi}$は
  \begin{equation}
    \ket{\xi} = \sum_{n\geq 0} C_{n}\ket{\Psi_{n}}
  \end{equation}
  と展開される。このとき、$\tau>0$として$e^{-\hat{H}\tau/\hbar}\ket{\xi}$という状態を考えると、
  \begin{equation}
    e^{-\hat{H}\tau/\hbar}\ket{\xi} = \sum_{n\geq 0} C_{n}e^{-E_{n}\tau/\hbar}\ket{\Psi_{n}} = e^{-E_{0}\tau/\hbar}\qty(C_{0}\ket{\Psi_{0}}+\sum_{n\geq 1} C_{n}e^{-(E_{n}-E_{0})\tau/\hbar}\ket{\Psi_{n}})
  \end{equation}  
  と表される。$\tau\to\infty$のとき、$C_{0}\ket{\Psi_{0}}$の項のみが残り、基底状態を得ることができる。
\end{frame}


\appendix

\end{document}