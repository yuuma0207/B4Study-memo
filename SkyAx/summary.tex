%ctrl+alt+m:open Math Preview
\documentclass[a4paper,11pt,uplatex]{jsarticle}%titlepage
%:/usr/local/texlive/texmf-local/tex/latex/report/report.sty
\usepackage{myreport}
\title{SkyAxの実装のためのメモ}
\author{20B01392 松本侑真}
\date{\today}
\begin{document}
\maketitle
\begin{abstract}
  SkyAxでは、平均場理論を用いて原子核の基底状態における様々な性質を計算することができる。
  SkyAxはHartree-Fock方程式を軸対称性を課した2次元の空間自由度の元で計算するプログラムである。
  本稿では、SkyAxの実装にあたってのメモを記す。
\end{abstract}
\tableofcontents
\newpage
\section{The Skyrme-Hartree-Fock theory}
\subsection*{The single-particle basis}
平均場理論では、1粒子波動関数$\psi_\alpha$と、占有確率振幅$v_{\alpha}$の組
\begin{equation}
  \{\psi_a,\,v_\alpha,\quad\alpha=1,\ldots,\,\Omega\}
\end{equation}
によって、原子核の基底状態を記述する。ここで、$\Omega$は1粒子波動関数と、占有確率振幅$0\leq v_\alpha\leq 1$の個数である。
非占有確率は$u_\alpha=\sqrt{1-v^2_\alpha}$と定義される。BCSの多体状態は、
\begin{equation}
  \ket{\Phi}=\prod_{\alpha>0}(u_\alpha+v_\alpha a^\dagger_\alpha a^{\dagger}_{\bar{\alpha}})\ket{0}
\end{equation}
と構成される。ここで、$\bar{\alpha}$は$\alpha$の時間反転対称な状態である。$a^\dagger_\alpha$は、$\psi_\alpha$の状態にあるフェルミオンの生成演算子である。

\subsection{Local densities and currents}
Skyrme-energy-density functionalは、いつかの局所的な密度と流れに依存する。これらの密度と流れは、1粒子波動関数$\psi_\alpha$と、占有確率振幅$v_\alpha$の組によって決定される。
今回は、時間反転対称性を課しているため、time-oddな項(current, spin density)は無視する。
密度と流れは、
\begin{align}
  \rho_\alpha(\bm{r})   & =\sum_{\alpha\in q}\sum_{s}v_\alpha^2\abs{\psi_\alpha(\bm{r},s)}^2                                                  &  & \text{density}                \\
  \bm{J}_\alpha(\bm{r}) & =-i\sum_{\alpha\in q}\sum_{ss'}v_\alpha^2\psi^*_\alpha(\bm{r},s)\nabla\cross \bm{\sigma}_{ss'}\psi_\alpha(\bm{r},s) &  & \text{spin-orbit density}     \\
  \tau_q(\bm{r})        & = \sum_{\alpha\in q}\sum_{s}v_\alpha^2\abs{\nabla\psi_\alpha(\bm{r},s)}^2                                           &  & \text{kinetic energy density} \\
  \xi_q(\bm{r})         & = \sum_{\alpha\in q}\sum_{s}w_\alpha u_\alpha v_\alpha \psi_{\bar{\alpha}}(\bm{r},s)\psi_\alpha(\bm{r},s)           &  & \text{pairing density}
\end{align}
と定義される。$w_\alpha$は対相関のsoft cutoffであり、$q$は原子核の種類のラベルである。$q=p$は陽子、$q=n$は中性子である。
変数$s\in \pm 1$は波動関数のスピノル部分である。今後、全密度や全流束をよく用いるが、これは核子のラベルがない物理量で表す。
例えば、全局所密度は
\begin{equation}
  \rho = \rho_p + \rho_n
\end{equation}
である。

\section*{Skyrme-Hartree-Fockの実装について}
Skyrme-Hartree-Fockは以下の式である:
\begin{align}
  \hat{h}\psi_\alpha & = \varepsilon_\alpha\psi_\alpha                                                      \\
  \hat{h}_q          & = U_q(\bm{r})-\div\qty[B_q(\bm{r})\grad]+i\bm{W}_q\cdot\qty(\bm{\sigma}\cross \grad)
\end{align}
$\hat{h}_q$の内訳として、
\begin{align}
  U_q(\bm{r}) = & b_0\rho - b'_0\rho_q + b_1\tau - b'_1\tau_q - b_2\Delta\rho + b'_2\Delta\rho_q + b_3\frac{\gamma+2}{3}\rho^{\gamma+1}-b'_3\frac{2}{3}\rho^{\gamma}\rho_q \notag                                                                   \\
                & - b'_3\frac{\gamma}{3}\rho^{\gamma-1}\sum_{q'}\rho_{q'}^2 - b_4\div{\bm{J}}-b'_4\div{\bm{J}_q} + \delta_{pq}\qty(U_C\frac{e^2}{2}\int dV'\,\frac{\rho_p(\bm{r}')}{\abs{\bm{r}-\bm{r}'}}-e^2\qty(\frac{3}{\pi})^{1/3}\rho_p^{1/3})
\end{align}
\begin{equation}
  B_q(\bm{r}) = \frac{\hbar^2}{2m_q}+b_1\rho -b'_1\rho_q
\end{equation}
\begin{equation}
  \bm{W}_q(\bm{r}) = b_4\grad\rho + b'_4\grad\rho_q - 2c_1\bm{J} + 2c'_1\bm{J}_q
\end{equation}
である。

\section{コードの実装のために}
\subsection{spin-orbit density}
spin-orbit densityの$ss'$の和について考える。まず、
\begin{equation}
  \curl{\bm{\sigma}_{ss'}} =
  \begin{pmatrix}
    \sigma_{ss'}^z\pdv{}{y} - \sigma_{ss'}^y\pdv{}{z} \\
    \sigma_{ss'}^x\pdv{}{z} - \sigma_{ss'}^z\pdv{}{x} \\
    \sigma_{ss'}^y\pdv{}{x} - \sigma_{ss'}^x\pdv{}{y}
  \end{pmatrix}
\end{equation}
\begin{equation}
  \sum_{ss'}\psi^*_{\alpha}(\bm{r},s)\curl{\bm{\sigma}_{ss'}\psi_\alpha(\bm{r},s)} =
\end{equation}

\section{式変形など}

\begin{equation}
  \hat{h}_q = U_q(\bm{r}) - \div\qty[B_q(\bm{r})\grad] + i\bm{W}_q\cdot\qty(\bm{\sigma}\cross \grad)
\end{equation}
$U_q$を求めるためには$\div{\bm{J}}_q$の計算が必要。そのための計算は波動関数の1階微分のみで計算できる。
$m_{\alpha}^{(-)}=m_{\alpha}^{(+)}+1 = m_{\alpha}+1$として、

\begin{align}
  \div{\bm{J}}_q                   & = \sum_{\alpha\in q}(\bm{\sigma}\cdot\grad\psi_\alpha)^{\dagger}(\bm{\sigma}\cdot\grad\psi_\alpha) - \tau_q                                                                                                                                              \\
  \bm{\sigma}\cdot\grad\psi_\alpha & =
  \begin{pmatrix}
    \grad_z\psi_\alpha^{(+)} + (\grad_r + m_\alpha^{(-)}/r)\psi_\alpha^{(-)} \\
    \qty[-\grad_z\psi_\alpha^{(-)} + (\grad_r - m_\alpha^{(+)} /r)\psi_\alpha^{(+)}]e^{i\phi}
  \end{pmatrix}e^{im\phi}                                                                                                                                                                      \\
  \tau_q                           & = \sum_{\alpha\in q}\qty(\grad\psi_\alpha)^{\dagger}\cdot\qty(\grad\psi_\alpha) = \sum_{\alpha,\pm}\qty{\abs{\grad_z\psi_{\alpha}^{(\pm)}}^2 + \abs{\grad_r\psi_{\alpha}^{(\pm)}}^2 + \qty(\frac{m_{\alpha}^{(\pm)}}{r})^2\abs{\psi_{\alpha}^{(\pm)}}^2}
\end{align}
\newpage

\subsection*{パウリ行列との外積}
\begin{align}
  (\curl\bm{\sigma})_r\psi_{\alpha} = (\grad_{\phi}\sigma_z - \grad_{z}\sigma_{\phi})\psi_{\alpha}
   & = \frac{1}{r}
  \begin{pmatrix}
    \partial_{\phi}\psi^{+}e^{im\phi} \\
    -\partial_{\phi}\psi^{-}e^{i(m+1)\phi}
  \end{pmatrix}
  -\begin{pmatrix}
     \partial_{z}(-i\psi^{-}e^{im\phi}) \\
     \partial_{z}(i\psi^{+}e^{i(m+1)\phi})
   \end{pmatrix} \\
   & =
  \begin{pmatrix}
    i\frac{m}{r}\psi^{+}e^{im\phi}  + i\partial_z \psi^{-}e^{im\phi} \\
    -i\frac{m+1}{r}\psi^{-}e^{i(m+1)\phi} - i\partial_z \psi^{+}e^{i(m+1)\phi}
  \end{pmatrix} \\
  &= 
  \begin{pmatrix}
    i(\frac{m}{r}\psi^{+} + \partial_z \psi^{-})e^{im\phi} \\
    -i(\frac{m+1}{r}\psi^{-} + \partial_z \psi^{+})e^{i(m+1)\phi}
  \end{pmatrix}
\end{align}
\begin{align}
  (\curl\bm{\sigma})_{\phi}\psi_{\alpha} = (\grad_{z}\sigma_r - \grad_{r}\sigma_{z})\psi_{\alpha}
   & =
  \begin{pmatrix}
    \partial_{z}\psi^{-}e^{im\phi} \\
    \partial_{z}\psi^{+}e^{i(m+1)\phi}
  \end{pmatrix}
  -\begin{pmatrix}
     \partial_{r}(\psi^{+}e^{im\phi}) \\
     \partial_{r}(-\psi^{-}e^{i(m+1)\phi})
   \end{pmatrix} \\
   & =
  \begin{pmatrix}
    \partial_z \psi^{-}e^{im\phi} - \partial_r \psi^{+}e^{im\phi} \\
    \partial_z \psi^{+}e^{i(m+1)\phi} + \partial_r \psi^{-}e^{i(m+1)\phi}
  \end{pmatrix} \\
  &= 
  \begin{pmatrix}
    (\partial_z \psi^{-}- \partial_r \psi^{+})e^{im\phi} \\
    (\partial_z \psi^{+} + \partial_r \psi^{-})e^{i(m+1)\phi}
  \end{pmatrix}
\end{align}
\begin{align}
  (\curl\bm{\sigma})_{z}\psi_{\alpha} = (\grad_{r}\sigma_{\phi} - \grad_{\phi}\sigma_{r})\psi_{\alpha}
   & =
  \begin{pmatrix}
    \partial_{r}(-i\psi^{-}e^{im\phi}) \\
    \partial_{r}(i\psi^{+}e^{i(m+1)\phi})
  \end{pmatrix}
  -\frac{1}{r}
  \begin{pmatrix}
    \partial_{\phi}\psi^{-}e^{im\phi} \\
    \partial_{\phi}\psi^{+}e^{i(m+1)\phi}
  \end{pmatrix} \\
   & =
  \begin{pmatrix}
    -i(\partial_r + m/r)\psi^{-}e^{im\phi} \\
    i(\partial_r - (m+1)/r)\psi^{+}e^{i(m+1)\phi}
  \end{pmatrix}
\end{align}

すなわち、$A,\,B\in \mathbb{R}$として、$r,\,z$成分は
\begin{equation}
  (\curl{\bm{\sigma}})_{r,\,z}\psi_{\alpha} = 
  \begin{pmatrix}
    iAe^{im\phi} \\
    iBe^{i(m+1)\phi}
  \end{pmatrix}
\end{equation}
の形になっており、$\phi$成分のみ
\begin{equation}
  (\curl{\bm{\sigma}})_{\phi}\psi_{\alpha} = 
  \begin{pmatrix}
    Ae^{im\phi} \\
    Be^{i(m+1)\phi}
  \end{pmatrix}
\end{equation}
の形になっている。
したがって、
\begin{align}
  \psi^{\dagger}_{\alpha}(\curl{\bm{\sigma}})_{\phi}\psi_{\alpha} &= 
  \begin{pmatrix}
    \psi^*_{\alpha,\uparrow}(r,z)e^{-im\phi} & \psi^*_{\alpha,\downarrow}(r,z)e^{-i(m+1)\phi}
  \end{pmatrix} 
  \begin{pmatrix}
    Ae^{im\phi} \\
    Be^{i(m+1)\phi}
  \end{pmatrix} \\
  &= \psi_{\alpha,\uparrow}A + \psi_{\alpha,\downarrow}B + \psi_{\alpha,\uparrow}Be^{i\phi} + \psi_{\alpha,\downarrow}Ae^{-i\phi}
\end{align}
となる。
\begin{equation}
  \bm{J}_q(\bm{r}) = \frac{1}{2i}\sum_{\alpha\in q}\qty(\psi^{\dagger}_{\alpha}(\curl{\bm{\sigma}})\psi_{\alpha} - \psi^{\top}_{\alpha}(\curl{\bm{\sigma}})\psi_{\alpha}^*) 
  = \Im{\sum_{\alpha\in q}\qty(\psi^{\dagger}_{\alpha}(\curl{\bm{\sigma}})\psi_{\alpha})}
\end{equation}
であるため、計算をすると
\begin{align}
  J_{r,z} &= \text{Im} + \qty(\psi_{\uparrow}B + \psi_{\downarrow}A)\cos\phi \\
  J_{\phi} &= \qty(\psi_{\uparrow}B + \psi_{\downarrow}A)\sin\phi
\end{align}
となる。

\newpage
\section{初期波動関数の配置方法}

原子核の半径$R$は核子数$A$について、$r_0=\SI{1.2}{fm}$とすると
\begin{equation}
  R=r_0A^{1/3}
\end{equation}
と表される。したがって、各核子の中心位置$(r_i,\,z_i)$を半径$R$の円内にランダムに配置するには、
\begin{equation}
  r_i = R\sqrt{v_i}\cos{(2\pi u_i)},\quad z_i = R\sqrt{v_i}\sin{(2\pi u_i)}
\end{equation}
とすればよい。ここで、$u_i,\,v_i$は$[0,\,1]$の一様乱数である。
次に、波動関数の形を決定する。ガウス型の波動関数を用いて、
\begin{equation}
  \psi_i = \exp(-\frac{1}{2a_0^2}\qty[(r-r_i)^2-(z-z_i)^2])
\end{equation}
と表す。$a_0$は波動関数の広がりを表すパラメータになる。

\newpage
\subsection{Nillsonモデルについて}
軸対称の調和振動子模型のハミルトニアンは
\begin{equation}
  H = -\frac{\hbar^2}{2\mu}\qty(\pdv[2]{r} + \frac{1}{r}\pdv{r} + \pdv[2]{z} + \frac{1}{r^2}\pdv[2]{\phi}) + \frac{1}{2}\mu\omega^2\qty(r^2 + z^2)
\end{equation}
と表される。これを円筒座標系で考えると、$\alpha = \sqrt{\mu\omega/\hbar}$として一般解が
\begin{align}
  R_{n_r}(r)   & = N_{n_r}r^{|m|}e^{-\alpha^2r^2/2}L_{n_r}^{|m|}(\alpha^2r^2) \\
  Z_{n_z}(z)   & = N_{n_z}e^{-\alpha^2z^2/2}H_{n_z}(\alpha z)                 \\
  \Phi_m(\phi) & = \frac{1}{\sqrt{2\pi}}e^{im\phi}
\end{align}
として、
\begin{equation}
  \Psi_{n_r,\,n_z,\,m} = R_{n_r}(r)Z_{n_z}(z)\Phi_m(\phi)
\end{equation}
と表される。$L_n^m$はLaguerre多項式、$H_n$はHermite多項式である。$N_{n_r},\,N_{n_z}$は規格化定数である。
Hermite多項式の規格化は
\begin{equation}
  \int_{-\infty}^{\infty}e^{-x^2}H_n(x)H_m(x)dx = \sqrt{\pi}2^nn!\delta_{nm}
\end{equation}
である。したがって、$Z_{n_z}(z)$の規格化条件は
\begin{equation}
  \int_{-\infty}^{\infty}\abs{N_{n_z}e^{-\alpha^2z^2/2}H_{n_z}(\alpha z)}^2dz = N_{n_z}^2\sqrt{\pi}2^{n_z}n_z! = 1
\end{equation}
となる。また、Hermite多項式の漸化式は
\begin{equation}
  H_{n+1}(x) = 2xH_n(x) - 2nH_{n-1}(x)
\end{equation}
である。これを用いると、
\begin{align}
  Z_{n_z+1}(z) & = N_{n_z+1}e^{-\alpha^2z^2/2}H_{n_z+1}(\alpha z) = N_{n_z+1}e^{-\alpha^2z^2/2}\qty(2\alpha zH_{n_z}(\alpha z) - 2n_zH_{n_z-1}(\alpha z)) \notag \\
               & = \frac{N_{n_z+1}}{N_{n_z}}\times2\alpha z\times Z_{n_z}(z) - \frac{N_{n_z+1}}{N_{n_z-1}}\times 2 n_{z}\times Z_{n_z-1}(z) \notag               \\
               & = \sqrt{\frac{2}{n+1}}\alpha z Z_{n_z}(z) - \sqrt{\frac{n_z}{n_z+1}}Z_{n_z-1}(z)
\end{align}
を得る。これで、すべての$n_z$についての波動関数が求まる。



また、Laguerre多項式の規格化は
\begin{equation}
  \int_{0}^{\infty}e^{-x}x^{m}L_n^m(x)L_m^m(x)dx = \frac{(n+m)!}{n!}\delta_{nm}
\end{equation}

\subsection*{動径方向について}
動径方向の波動関数は、$\Phi_m(\phi) \propto e^{im\phi}$なので
\begin{equation}
  -\frac{\hbar^2}{2\mu}\qty(\pdv[2]{r} + \frac{1}{r}\pdv{r} - \frac{m^2}{r^2})R_{n_r}(r) + \frac{1}{2}\mu\omega^2r^2R_{n_r}(r) = E_{n_r}R_{n_r}(r)
\end{equation}
となる。$x=\alpha r$と変数変換をすると
\begin{equation}
  \qty[-\qty(\pdv[2]{x} + \frac{1}{x}\pdv{x} - \frac{m^2}{x^2}) + x^2]R_{n_r}(x) = \frac{2E_{n_r}}{\hbar\omega}R_{n_r}(x)
\end{equation}
となる。
\begin{equation}
  R_{n_r}(x) = \frac{u_{n_r}(x)}{x}
\end{equation}
とおくと、
\begin{equation}
  \qty[\dv[2]{x} - \qty(\frac{m^2}{x^2} + x^2 - \frac{2E_{n_r}}{\hbar\omega})]u_{n_r}(x) = 0
\end{equation}
を得る。この一般解は
\begin{equation}
  u(x) = x^{m+1}
\end{equation}
動径方向わからない
\end{document}





