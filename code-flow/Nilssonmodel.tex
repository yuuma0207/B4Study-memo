%ctrl+alt+m:open Math Preview
\documentclass[a4paper,11pt,uplatex]{jsarticle}%titlepage
%:/usr/local/texlive/texmf-local/tex/latex/report/report.sty
\usepackage{myreport}
\title{Nilsson modelの実装についてのメモ}
\author{20B01392 松本侑真}
\date{\today}
\begin{document}
\maketitle
\begin{abstract}

\end{abstract}
\tableofcontents
\newpage
ハミルトニアンは
\begin{equation}
  H = -\frac{\hbar^2}{2m}\laplacian + \frac{m}{2}\qty(\omega_r^2(x^2+y^2) + \omega_z^2z^2)
\end{equation}
となり、固有エネルギーは
\begin{equation}
  E = \hbar\omega_r(n_x+n_y+1) + \hbar\omega_z(n_z+1/2)
\end{equation}
である。ここで、$n_x,\,n_y,\,n_z$はそれぞれ$x,\,y,\,z$方向の量子数である。
ラプラシアンは円筒座標$(r=\sqrt{x^2+y^2},\phi,z)$で表すと
\begin{equation}
  \laplacian = \frac{1}{r}\pdv{r}\qty(r\pdv{r})+ \frac{1}{r^2}\pdv[2]{\phi} + \pdv[2]{z}
\end{equation}
であるため、
\begin{equation}
  \qty[-\frac{\hbar^2}{2m}\qty(\frac{1}{r}\pdv{r}\qty(r\pdv{r})+ \frac{1}{r^2}\pdv[2]{\phi} + \pdv[2]{z}) + \frac{m}{2}\qty(\omega_r^2r^2 + \omega_z^2z^2)]\psi = E\psi
\end{equation}
となる。
$\psi = R(r)Z(z)\Phi(\phi)$と変数分離をすると、
\begin{equation}
  -\frac{\hbar^2}{2m}\qty(\frac{Z(z)\Phi(\phi)}{r}\pdv{r}\qty(r\pdv{R(r)}{r})+ \frac{R(r)Z(z)}{r^2}\pdv[2]{\Phi(\phi)}{\phi} + R(r)\Phi(\phi)\pdv[2]{Z(z)}{z}) 
  + \qty[\frac{m}{2}\qty(\omega_r^2r^2 + \omega_z^2z^2)]\psi = E\psi
\end{equation}
となり、両辺$\psi$で割ると
\begin{equation}
  -\frac{\hbar^2}{2m}\qty(\frac{1}{r}\pdv{r}\qty(r\pdv{R(r)}{r})\frac{1}{R(r)} + \frac{1}{r^2}\pdv[2]{\Phi(\phi)}{\phi}\frac{1}{\Phi(\phi)} + \pdv[2]{Z(z)}{z}\frac{1}{Z(z)}) 
  + \qty[\frac{m}{2}\qty(\omega_r^2r^2 + \omega_z^2z^2)] = E
\end{equation}
となる。定数$m_{\phi}$を用いて書き換えると
\begin{equation}
  \frac{\Phi''}{\Phi} = -\frac{2mr^2}{\hbar^2}\qty(E -\qty[\frac{m}{2}\qty(\omega_r^2r^2 + \omega_z^2z^2)] +\frac{\hbar^2}{2m}\qty(\frac{1}{r}\pdv{r}\qty(r\pdv{R(r)}{r})\frac{1}{R(r)} + \pdv[2]{Z(z)}{z}\frac{1}{Z(z)})) = -m_{\phi}^2
\end{equation}
となる。これより$\Phi(\phi)$の解
\begin{equation}
  \Phi(\phi) = A\exp(im_{\phi}\phi)
\end{equation}
と、$R(r),\,Z(z)$についての方程式
\begin{equation}
  -\frac{\hbar^2}{2m}\qty[\frac{1}{r}\pdv{r}\qty(r\pdv{R(r)}{r})\frac{1}{R(r)} + \pdv[2]{Z(z)}{z}\frac{1}{Z(z)}] = E -\qty[\frac{m}{2}\qty(\omega_r^2r^2 + \omega_z^2z^2)]
\end{equation}
を得る。$z$方向は1次元調和振動子の方程式であるため、解はエルミート多項式で表される:
\begin{equation}
  Z(z) = H_n(\sqrt{\frac{m\omega_z}{\hbar}}z)
\end{equation}

\end{document}