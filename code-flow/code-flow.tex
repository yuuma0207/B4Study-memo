\documentclass[a4paper,11pt,titlepage,uplatex]{jsarticle}
\usepackage{myreport}
\renewcommand{\thesection}{1-\arabic{section}}
\title{SkyAxの実装についての流れ}
\author{20B01392 松本侑真}
\date{\today}
\begin{document}
\maketitle
\begin{abstract}

\end{abstract}
\tableofcontents
\newpage

\section{ハミルトニアンの計算}
波動関数を
\begin{equation}
    \psi_{\alpha} = 
    \begin{pmatrix}
      \psi_{\alpha}^{+} \\
      \psi_{\alpha}^{-}
    \end{pmatrix}
     = 
    \begin{pmatrix}
      f_{\alpha}^{+}e^{im\phi} \\
      f_{\alpha}^{-}e^{i(m+1)\phi}
    \end{pmatrix}
\end{equation}
とおく。ポテンシャル
\begin{equation}
  U,\quad \bm{W},\quad B
\end{equation}
が計算できれば、ハミルトニアンが求まる。
$W$に関係する項(W\_term)は
\begin{equation}
  i\bm{W}\cdot(\bm{\sigma}\cross\nabla)
\end{equation}
となっている。ここで、
\begin{align}
  (\bm{\sigma}\cross\nabla)_r\psi_{\alpha} &=
  (\sigma_{\phi}\grad_{z} - \sigma_{z}\grad_{\phi})\psi_\alpha 
  = 
  \begin{pmatrix}
  -ie_-\grad_{z}  \psi_{-} - \grad_{\phi}\psi_{+} \\
  ie_-\grad_{z}\psi_{+} + \grad_{\phi}\psi_{-}
  \end{pmatrix}
  =
  \begin{pmatrix}
    -i\partial_{z}f_{-} - i\frac{m}{r}f_{+} \\
    +i\partial_{z}f_{+}  + i\frac{m+1}{r}f_{-}
  \end{pmatrix} \\
  (\bm{\sigma}\cross\nabla)_\phi\psi_{\alpha} &=
  (\sigma_{z}\grad_{r} - \sigma_{r}\grad_{z})\psi_\alpha 
  = 
  \begin{pmatrix}
   \grad_{r}\psi_{+} - e_{-}\grad_{z}\psi_{-} \\
  -\grad_{r}\psi_{-} - e_{+}\grad_{z}\psi_{+}
  \end{pmatrix}
  =
  \begin{pmatrix}
    \partial_{r}f_{+} - \partial_{z}f_{-} \\
   -\partial_{r}f_{-} - \partial_{z}f_{+}
  \end{pmatrix} \\ 
  (\bm{\sigma}\cross\nabla)_z\psi_{\alpha} &=
  (\sigma_{r}\grad_{\phi} - \sigma_{\phi}\grad_{z})\psi_\alpha 
  = 
  \begin{pmatrix}
   e_{-}\grad_{\phi}\psi_{-} + ie_{-}\grad_{r}\psi_{-} \\
   e_{+}\grad_{\phi}\psi_{+} - ie_{+}\grad_{r}\psi_{+}
  \end{pmatrix}
  =
  \begin{pmatrix}
   i\frac{m+1}{r}f_{-} + i\partial_{r}f_{-} \\
   i\frac{m}{r}  f_{+} - i\partial_{r}f_{+}
  \end{pmatrix}
\end{align}

\subsection{Poisson方程式}
ときたいものは
\begin{equation}
  \pdv{r}\qty(r\pdv{f}{r}) + \pdv[2]{f}{z}  = \rho
\end{equation}
つまり
\begin{equation}
  \pdv[2]{f}{r} + \frac{1}{r}\pdv{f}{r} + \pdv[2]{f}{z} = 0
\end{equation}
中心差分でやると
\begin{equation}
  \frac{1}{r}\pdv{f}{r} = \frac{f_{i+1,j}-f_{i-1,j}}{2r\Delta r}
\end{equation}
\begin{equation}
  \pdv[2]{f_{ij}}{r} = \frac{f_{i+1,j}-2f_{i,j}+f_{i-1,j}}{(\Delta r)^2}
\end{equation}
\begin{equation}
  \pdv[2]{f_{ij}}{z} = \frac{f_{i,j+1}-2f_{i,j}+f_{i,j-1}}{(\Delta z)^2}
\end{equation}
となる。よって、
\begin{equation}
  \frac{f_{i+1,j}-2f_{i,j}+f_{i-1,j}}{(\Delta r)^2} 
  +\frac{f_{i+1,j}-f_{i-1,j}}{2r\Delta r}
  +\frac{f_{i,j+1}-2f_{i,j}+f_{i,j-1}}{(\Delta z)^2} = \rho_{ij}
\end{equation}
となる。
\begin{equation}
  \qty[\frac{1}{(\Delta z)^2}]f_{i,j-1}+\qty[\frac{1}{(\Delta r)^2}-\frac{1}{2r\Delta r}]f_{i-1,j} + \qty[\frac{-2}{(\varDelta r)^2} + \frac{-2}{(\varDelta z)^2}]f_{i,j}
  + \qty[\frac{1}{(\varDelta r)^2} + \frac{1}{2r(\varDelta r)}]f_{i+1,j} + \qty[\frac{1}{(\varDelta z)^2}]f_{i,j+1} = \rho_{ij}
\end{equation}
となり、これは
\begin{equation}
  A_{2}f_{i,j-1}+A_{1}^{-}f_{i-1,j} + A_0f_{i,j} + A_{1}^{+}f_{i+1,j} + A_{2}f_{i,j+1} = \rho_{ij}
\end{equation}
となる。また、$i,j\to k$にマッピングするとき、$k$が与えられると
\begin{align}
  i &= \mathrm{mod}\,(k-1,N)+1 \\
  j &= [k/N]+1 
\end{align}
と置くことができる。また、$(i,j)$が与えられると
\begin{equation}
  k = (j-1)N + i
\end{equation}
と与えられる。
\subsection{境界条件}
$r=0$の境界条件として、偶関数であることを要求する。また、原点の値は、$U(1,j)$と同じものとする。
$r=Nr*dr$の境界条件は固定端境界条件とする。つまり、
\begin{align}
  U(-i,j) &= U(i,j) \\
  U(0,j)  &= U(1,j) \\
  U(Nr,j) &= 0 
\end{align}
とすればよい。また、$z=0$の境界条件は固定端である:
\begin{align}
  U(i,0) &= 0 \\
  U(i,Nz) &= 0
\end{align}


\end{document}